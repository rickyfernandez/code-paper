\subsubsection{Self-Similar infall test}
\label{sec.tests.infall}

This test problem is based on the self-similar solutions to the
cosmological spherical infall problem found by
\citet{Bertschinger1985}.  It features a small density perturbation in
a homogeneous $\Omega = 1$ universe and is a strong test of the
cosmological evolution, gravity, and hydrodynamic portions of the
code.  It is a close analogue to cosmological halo formation.

For the initial conditions, we adopt a $32^3$ top grid with a single,
initial subgrid (covering $1/8^3$ of the domain) with a refinement
factor of 2.  An overdensity $\delta = 40$ is placed in a single cell
near the center of the domain.  We begin at $z=199$ and evolve to
$z=0$, which is a sufficient time for the evolution to largely forget
its initial conditions and approach the self-similar result.  An
overdensity refinement criteria ($\delta_{\rm crit} = 1.1$ on the top
grid) is used to add additional grids, going up to 5 additional levels
beyond the root grid.  We use the PPM solver without radiative
cooling; only baryons are used for this problem.  An ideal gas law
with $\gamma = 5/3$ is adopted.

In Figure~\ref{fig.sphericalinfall}, we show the results, scaled
according to the dimensionless variables as defined in equations (2.9)
and (3.2) of \citet{Bertschinger1985}.  This demonstrates that we can
quickly and easily obtain a good solution with only a fairly modest
initial grid.  The shock is sharply resolved and the asymptotic
profiles at small $\lambda$ are recovered.  At very small values of
$\lambda$, the initial conditions have not been fully forgotten and
the self-similar solution is not recovered.  This can be seen most
clearly in the dimensionless mass.  However, this is simply because of
the limited amount of time for which we evolve the solution.


\begin{figure}
\begin{center}
\includegraphics[width=0.7\textwidth]{figures/SphericalInfall.eps}
\caption{The results of the self-similar spherical infall test.  The
panels show, as a function of the dimensionless radius $\lambda$, the
dimensionless density $D$ (top left), velocity $V$ (top right),
pressure $P$ (bottom left), and enclosed mass $M$ (bottom right).  In
each case we show azimuthally averaged profiles from the simulation as
circles and the analytic solution for $\gamma = 5/3$ as solid lines.}
\label{fig.sphericalinfall}
\end{center}
\end{figure}

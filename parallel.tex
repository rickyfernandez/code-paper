\section{Parallel Strategy and Performance}
\label{sec.parallel}

\subsection{Parallel Strategy}

The current version of \enzo\ has been parallelized for distributed
memory platforms using the Message Passing Interface (MPI).  This is
done using a single grid object as the basic unit of parallelization.
Each grid object -- including all cell and grid data -- is fully
contained on a single processor\footnote{In this context, we use the
word {\it processor} to mean a basic distribution unit; this could be
either a core or a node, depending on details of the system.  Note
that the current version of \enzo\ is not threaded, although a hybrid
MPI + OpenMP version is under development.}.  Parallelization is
accomplished by distributing grids amongst processors.  This is done
on the root grid using a simple tiling system, where the root grid is
split up into $N_{\rm root}$ tiles, with $N_{\rm root}$ typically
equal to the number of processors, $N_p$.

Load balancing on levels other than the root level (i.e., grids for
which the level $l > 0$) is different, as the refined patches are not
generally uniformly distributed.  Grids on refined levels are first
placed on the same processor as their parent to minimize
communication; however, this generally does not result in a
well-balanced computational load.  Therefore, the code has a number of
options for load balancing the grids on a given level. Each grid is
assigned an estimated computational load, generally equal to the
number of cells in the grid (which, empirically, is a good estimate of
computational cost).  The first load-balancing option is to move a
grid from the processor with the highest computational load to the
processor with the lowest load, with the proviso that only grids with
load factors less than half the difference between the highest and
lowest loaded processors will be moved.  This continues until the load
ratio between the most-to-least loaded processor is below 1.05 or
until no suitable grid can be found to transfer.  A second load
balancing option uses a space-filling Hilbert curve to order the grids
by their approximate spatial position.  Then, once the grids have a
specific one-dimensional ordering, we can divide up the grids into
$N_p$ groups (with the division taking place as equally as possible).
Load balancing is done separately for each level.  Clearly load
balancing is most successful if there are significantly more grids
than processors; however, small grids are less efficient because of
the large number of ghost zones compared to active zones, and so the
code uses a simple heuristic in order to split up grids until there
are of order 10 grids per processor.  This generally results in good
load-balancing while not producing grids that are wastefully small.

Communication between processors is done using a non-blocking
communication strategy that allows overlap of communication and
computation.  This can be done efficiently because each processor
retains a copy of the entire hierarchy of grids, except that grids
that do not `live' on a given processor only contain meta data
(essentially just the location and size of the grid; such grids are
denoted as `ghost' grids).  Replicating the hierarchy means that all
communication of data from one grid to another can be identified by
each processor independently.  The metadata for `ghost' grids are
quite small and so the extra memory required is generally not onerous
unless very large numbers of grids are used (more than a few hundred
thousand grids).  A schematic of this distribution is shown in
Figure~\ref{fig.amr_hierarchy}.

Data is transferred through a three-step procedure that takes
advantage of the capabilities of the MPI library: (i) as the code
progresses and data is needed from another grid on another processor,
the receiving processor posts an MPI non-blocking receive indicating
that it is expecting data; this outstanding receive is recorded in a
table, (ii) the sending processor calls the MPI non-blocking send
function, and then finally (iii) the receiving processor, after it has
carried out all the computation it can, waits for any MPI message to
arrive.  Each message is coded so that it can be matched with the
appropriate receive posted in the first step, and based on that, the
appropriate routine is called to processes the data.  Step (iii) is
repeated until there are no outstanding receives.

% ----------------------------------

\subsection{Performance}
\label{sec.performance}

\subsubsection{Performance Measurement \& Instrumentation}

Because of the wide variety of simulations, methods, and uses of
\enzo, it is difficult to state in general terms which routines within
the code will be most costly during a given simulation.  As such, we
have designed a lightweight registering system that has been
implemented for the most commonly used routines (such as the
hydrodynamic and gravity solvers) as well as refinement level timers
that measure the time spent on each level.  Beyond this minimal set of
routines, we have designed a simple way for the user to modify the
source by adding \texttt{TIMER\_START("Your\_Routine\_Name")} and
\texttt{TIMER\_END("Your\_Routine\_Name")}.  These timers are created
and managed individually on each processor in an asynchronous fashion,
and contribute minimal computational, memory, and IO overhead.

At each complete root grid timestep (or less often if specified), each
timer is then communicated to the root processor where it calculates
the mean, standard deviation, minimum, and maximum for each of the
timers of that name across all processors.  For level timers, there
are additional attributes such as the number of cell updates, the
current number of grids, and the average cells updates per second per
MPI process.  This information is then output to a logfile.  This
provides a simplified interface to the user that can be used to
diagnose performance issues as well as estimate a given problem type's
scalability.  In addition to the logfile, we have developed a plotting
interface for quickly producing figures that process the data from the
logfile.  These capabilities are described in the online
documentation, along with further discussion of the performance
measurement implementation.

\subsubsection{Unigrid scaling}
\label{sec:weak_scaling}

\begin{figure}
\begin{center}
\includegraphics[width=0.6\textwidth]{figures/enzo_unigrid_weak_scaling.eps}
\caption{\enzo\ weak scaling performance for a set of Lyman-$\alpha$
forest cosmology simulations with constant comoving spatial resolution
per grid cell, showing cell updates per second per processor plotted
as a function of the number of root grid tiles of dimension $128^3$
(R) in each dimension.  The number of MPI tasks is N$ = R^3$, so R$ =
16$ on this plot corresponds to a $2048^3$ computational mesh running
on 4,096 MPI tasks.  This plot goes from R$ = 2$ (8 MPI tasks) to R$ =
24$ (13,824 MPI tasks) on two supercomputers -- NICS Kraken and ORNL
Jaguar when they were Cray XT4 systems -- and using 1, 2 or 4 MPI
processes per node, where each compute node contained a single
quad-core AMD Opteron CPU with a speed of 2.1 GHz on Jaguar and 2.3
GHz on Kraken.}
\label{fig.uniscale}
\end{center}
\end{figure}

It is advantageous to use \enzo\ in its ``unigrid'' (i.e.,
non-adaptive mesh) mode for a variety of problems, including
non-gravitating turbulence
\citep[e.g.,][]{2002ApJ...569L.127K,Kritsuk04, 2007ApJ...665..416K,
2009ASPC..406...15K}, the Lyman-$\alpha$ forest
\citep{2005MNRAS.361...70J,2009MNRAS.399.1934P}, or feedback of
metal-enriched gas from galaxies into the intergalactic medium
\citep{2004ApJ...601L.115N,2011ApJ...731....6S}.  Achieving good
scaling of the code in unigrid mode is relatively straightforward --
upon initialization, unigrid simulations are decomposed such that each
MPI process has a roughly equal subvolume (and thus an equal number of
grid cells), meaning that work is evenly distributed among
computational elements.  Communication patterns for both the gravity
solve (which uses a fast Fourier transform) and the fluid solves
(which transfer boundary information between subvolumes) are
predictable and straightforward, and rebuilding of the grid hierarchy
does not take place, removing a substantial global operation and a
great deal of communication.

Figure~\ref{fig.uniscale} shows \enzo\ weak scaling results for a
sequence of scaled unigrid Lyman-$\alpha$ forest calculations. These
calculations include dark matter dynamics, hydrodynamics using the PPM
solver, six-species non-equilibrium chemistry and radiative cooling,
and a uniform metagalactic ultraviolet background.  In this sequence
of test calculations, we perform a weak scaling test on up to 13,824
MPI tasks on the NICS Kraken XT4 and ORNL Jaguar XT4
supercomputers\footnote{These simulations were performed prior to
conversion of both machines to the current-generation systems}.  In
this test, each MPI task was given a $128^3$ root grid tile (i.e.,
$128^3$ grid cells containing baryon quantities) and, initially,
approximately $128^3$ dark matter particles.  The number of grid cells
on each processor was constant throughout the calculation; the number
of dark matter particles on each processor varies as they are moved
from subvolume to subvolume as structure evolves.  The grid resolution
was kept at a constant comoving size of $\simeq 40$~kpc/h, and as the
core count was increased, so was the simulation volume.  On each
machine, a compute node contained a single AMD Opteron quad-core chip
(2.1 Ghz on Jaguar; 2.3 Ghz on Kraken) with 2 GB/memory per core (8
GB/total per node).  Both machines used the SeaStar2 interconnect.  In
the scaling study, calculations were run with 1, 2, or 4 MPI tasks per
node.  The figure shows cell updates per second per MPI process;
perfect scaling would be a horizontal line.

As can be seen in Figure~\ref{fig.uniscale}, the unigrid weak scaling
performance of the code is extremely good for this problem, with only
a 20\% decrease in cell updates per second per MPI task as the code is
scaled from 8 to 4,096 MPI tasks, and a 40\% decrease in performance
overall going from 8 to 13,824 (or $24^3$) MPI tasks.  We speculate
that this decrease is likely to be partially due to global MPI
communications used to, e.g., calculate the overall timestep of the
simulation, and also likely due to load imbalances due to increasing
cosmological power (and thus an increasingly uneven distribution of
dark matter particles between MPI tasks at late times) as the
simulation volume grows.  We also observe that a systematic difference
in speed can be seen between the two machines, which can be attributed
primarily to the slightly faster CPUs on Kraken at the time (2.3 Ghz,
vs. 2.1 Ghz on Jaguar).  The difference in speed when using different
numbers of MPI tasks per node can be attributed primarily due to
differences in competing usage of shared cache on the quad-core chips
used on this machine.

Broadly, excellent scaling in \enzo's unigrid mode is seen for a
variety of problems as long as each compute core is given an adequate
amount of work to do.  For cosmological simulations, this value has
been empirically determined to be roughly $128^3$ cells per core.  If
fewer cells per core are used, the CPU is essentially data-starved and
poor scaling is observed due to computing units being idle while
waiting for information to be communicated from other processes (for,
e.g., boundary information or gravity solves).  Substantially larger
cell counts per core would in principle help scaling by reducing the
amount of inter-process communication needed.

As a final point, we observe that scaling at larger core counts has
been measured, but only with an experimental hybrid-parallel (MPI +
OpenMP) version of \enzo.  Using this version, scaling comparable to
that shown in Figure~\ref{fig.uniscale} was seen on up to 98,304 cores
on the NICS Kraken XT5 (an upgraded version of the XT4 machine used
for the scaling study shown in the figure), using 2-8 OpenMP threads
per MPI process.  Hybrid parallelism has the potential to
substantially improve scaling by reducing the amount of communication
per grid tile, as described in the previous paragraph.

\subsubsection{AMR scaling}

\begin{figure}
\begin{center}
\includegraphics[width=0.48\textwidth]{figures/strong_scaling_levels.eps}
\hfill
\includegraphics[width=0.48\textwidth]{figures/strong_scaling_routines.eps}
\end{center}
\caption{\emph{Left:} Strong scaling test of a 512$^3$ AMR
cosmological calculation.  The root grid scaling is not representative
of the true strong scaling of \enzo\ because the root grid tiles are
not repartitioned when a simulation is restarted with additional
cores.  The weak scaling test in Figure \ref{fig.uniscale} is more
representative of the scaling on the root level.  The performance in
the refined levels show good strong scaling up to 2,048 cores.
\emph{Right:} Time spent in representative major routines in the same
AMR calculation.}
\label{fig:strong_scaling}
\end{figure}

Many astrophysical problems, such as cosmological galaxy formation
\citep{2012ApJ...749..140H}, high-resolution disk galaxy simulations
\citep{2011ApJ...738...54K}, high-redshift
\citep{2009Sci...325..601T}, and present-day star formation
\citep{Collins12a}, involve multi-scale physics that span several
orders of magnitude in both space and time.  In these situations,
using \enzo\ in its adaptive mesh refinement mode is beneficial.
Because the adaptive grid hierarchy is dynamic, grid boundaries and,
thus, communication patterns can be unpredictable, hindering strong
scaling to high core counts.

Figure \ref{fig:strong_scaling} shows strong scaling results from a
single 50 Mpc/h cosmology simulation run on $N_{\rm core}$ compute
cores, ranging from 128 to 16,384 cores at power-of-two intervals.  It
was run on the NICS Kraken XT5 supercomputer, which has two AMD
Opteron hexa-core processors with 16 GB of memory per compute node.
The simulations that utilized 128, 256, and 512 cores were executed on
128 nodes because of the memory requirements.  The higher core count
simulations were run with 8 cores per node.  This simulation would not
run with 12 cores per node because of the memory overhead associated
with the grid hierarchy being duplicated on each MPI process.
However, this overhead is greatly diminished if a hybrid-parallel (MPI
+ OpenMP) approach is used.

This simulation includes dark matter dynamics, hydrodynamics using the
piecewise parabolic method, six-species non-equilibrium chemistry and
radiative cooling, and a uniform metagalactic ultraviolet background.
The simulation uses the space-filling curve method for load balancing
the AMR grids.  It has a $512^3$ root grid that is divided into 512
tiles, and 6 additional AMR levels are used.  We perform these scaling
tests when the simulation reaches $z=4$, where the AMR grid hierarchy
is well-developed and is thus a reasonable representation of AMR
simulation behavior in general.  The results shown in Figure
\ref{fig:strong_scaling} come from a single root-level timestep of
$\Delta t = 2.1\, \textrm{Myr}$.  At this time, there are $3.04 \times
10^5$ AMR grids, $9.03 \times 10^8$ ($\sim 967^3$) computational
cells, and $1.34 \times 10^8$ dark matter particles in total.  The
breakdown of the number of AMR grids, cells, timesteps, and number of
cell updates on each level is shown in Table \ref{tab:amr_scale}.

%%%%%%%%%%%%%%%%%%%%%%%%%%%%%%%%%%%%%%%%%%%%%%%%%%%%%%%%%%%%%%%%%%%%%%%%
\begin{table*}
  \begin{center}
  \caption{Strong scaling test computational details}
  \begin{tabular*}{0.9\textwidth}{@{\extracolsep{\fill}}c c c c c}
    \tableline\tableline
    {Level} & {$N_{\rm grid}$} & {$N_{\rm cells}$} & {$N_{\rm up}$} &
    {$N_{\rm timesteps}$}\\
    \tableline
    0 & 512 & $1.34 \times 10^8$ & $1.34 \times 10^8$ & 1\\
    1 & 61,868 & $4.01 \times 10^8$ & $4.01 \times 10^8$ & 1\\
    2 & 91,182 & $1.99 \times 10^8$ & $5.96 \times 10^8$ & 3\\
    3 & 59,932 & $7.62 \times 10^7$ & $5.34 \times 10^8$ & 7\\
    4 & 40,700 & $3.32 \times 10^7$ & $5.65 \times 10^8$ & 17\\
    5 & 28,346 & $2.76 \times 10^7$ & $1.36 \times 10^9$ & 49\\
    6 & 19,771 & $2.80 \times 10^7$ & $5.25 \times 10^9$ & 187\\
    \tableline
    Total & 302,311 & $9.03 \times 10^8$ & $8.83 \times 10^9$ & --\\
  \end{tabular*}
  \parbox[t]{0.9\textwidth}{\textbf{Note.} --- Data shown at $z=4$ for
    a root grid timestep of 2.1~Myr.  
    Col. (1): AMR Level. Col. (2):
    Number of grids. Col. (3): Number of computational
    cells. Col. (4): Number of cell updates. Col. (5): Number of
    timesteps.}
  \label{tab:amr_scale}
  \end{center}
\end{table*}

%%%%%%%%%%%%%%%%%%%%%%%%%%%%%%%%%%%%%%%%%%%%%%%%%%%%%%%%%%%%%%%%%%%%%%%%

The left panel in Figure \ref{fig:strong_scaling} shows the
computational and communication time spent on each level.  In the AMR
levels, there exists good strong scaling up to 2,048 cores, and
marginal speed-ups are found at 4,096 cores.  On the root-grid level,
there exists good scaling up to 1,024 cores, but the performance
dramatically decreases at higher core counts.  This occurs because the
root grid is not re-partitioned into $N_{\rm core}$ tiles when the
simulation is restarted with a different core count.  This feature can
be easily implemented and is planned in the next major release of
\enzo, where scaling results would be similar to the weak scaling
shown in \S\ref{sec:weak_scaling}.  The right panel in Figure
\ref{fig:strong_scaling} shows the time spent in some representative
major routines in \enzo.  The local physics routines, for example
\texttt{SolveHydroEquations}, exhibit perfect strong scaling because
they involve no communication.  By investigating the scaling behavior
in each routine, it is clear that the communication in the
\texttt{SetBoundaryConditions}, \texttt{SolveForPotential} (the
multi-grid solver in AMR subgrids), and \texttt{RebuildHierarchy} are
responsible for the lack of strong scaling at $N_{\rm core} \ga$~4,096
in this particular simulation.  The transpose of the root grid tiles
are responsible for the performance decrease because it is not
optimized for situations where the number of tiles is greater than the
number of MPI processes.  These results are directly applicable to
simulations with similar computational demands.  In simulations with
fewer computational cells, strong scaling will cease at a smaller core
count because the CPUs will become data-starved more quickly, and the
opposite occurs with larger simulations.


\subsubsection{An Approximate Time and Memory Scaling Model}

In this section, we develop a simple, approximate model to estimate
the computational time and memory required to complete a given
calculation.  Note that we are {\it not} trying to model parallel
performance (which was discussed in the previous section), but have an
even simpler goal: to determine how to scale computational time
estimates as we increase the simulation resolution.  This is a
straightforward thing to do in codes that use static grids, as the
computational effort per timestep is constant.  However, for an AMR
calculation, the number of grids that will be generated during the run
is not known in advance, and so the CPU time per problem time can vary
drastically throughout the calculation.  For example, at the beginning
of a cosmological simulation, when the densities are nearly uniform,
only the static grid is required and the calculation progresses
rapidly.  However, as structure forms and dense clumps are generated,
the number of grid points swells by orders of magnitude (an increase
of $10^3$ is not uncommon) and most of the CPU time is consumed at
late physical time in the simulation.  Therefore, simply performing a
few steps at the beginning of the calculation does not produce a good
estimate of the required CPU time.

Nevertheless, we can try to determine how the compute time scales for
a given run as we increase the resolution.  First, we neglect
particles and concentrate on the time taken by the grids, which can be
justified both theoretically and empirically\footnote{Since the
potential is calculated on the grid, the only particle costs are:
depositing mass to the grid, interpolating accelerations, and updating
particle positions and velocities.  These are relatively
computationally inexpensive operations.}.  Second, we break the
problem down slightly, and examine the scaling over a short enough
period of time that the grid structure does not change significantly
(i.e. the number of grids at each level remains approximately
constant).  We then assume that the whole run scales in the same way,
or in other words, that changes in the resolution affect the
simulation in the same way at each timestep.  This is usually a good
approximation, as the most costly parts of the calculation typically
don't change their characteristics substantially between timesteps.

To make progress, we assume that the computational cost to advance a
single cell by one timestep is a constant.  This can be incorrect if
the chemistry solver requires many iterations, but is usually fairly
accurate.  For a unigrid calculation, the time would by proportional
to $N_{\rm root}^4$ since the number of cells scales as $N_{\rm
root}^3$ and, assuming the Courant condition is the factor that
controls the timestep, the number of steps to advance the calculation
over a given time interval is proportional to $N_{\rm root}.$
Therefore, to advance a hierarchy a given time interval, we find,
accounting for all levels and using a refinement factor of 2,

\begin{equation} 
t_{\rm SU} = C_1 \sum_l f_l N_{\rm root}^4 2^{4l}
\end{equation} 

where $C_1$ is a constant, which can be thought of as the time taken
to advance a single cell.  The factor $f_l$ is the fraction of the
volume on a given level that is actually refined.  By definition $f_0
= 1$, and $f_l \le 1$.  In writing down this equation, we neglect a
number of costs that are not directly proportional to cell count,
including communication between processors, the $\log{N}$ factor for
the root grid FFT, cache misses, optimizations, and other costs
associated with processing the hierarchy.  The first item on this
list, in particular, is clearly important for large processor counts;
however, we neglect parallel considerations in this section.

Note that unless $f_l/f_{l+1} < 1/16$ (i.e. if less then about 6\% of
a given level is further refined), the cost per level will increase
with level.  A key question, therefore, is the value of $f_l$ for each
level.  Unfortunately, this depends strongly on the simulation being
run.  In Figure~\ref{fig:scaling}, we show the values of $f_l$ for
three simulations: two cosmological runs with varying box size, and a
third simulation focusing on a single disk galaxy.

\begin{figure}
\centerline{\includegraphics[width=0.7\textwidth]{figures/scaling_plot2}}
\caption{This figure shows (clockwise from upper left), the covering
fraction $f_l$ of the grids on a given level $l$, the ratio of
$f_{l}/f_{l+1}$, the relative memory usage of each level (normalized
by the memory usage of level 0), and, in the lower-left panel, the
relative CPU usage of each level, again normalized by $l=0$.  In each
panel, three curves are shown: the solid black and long dashed red are
two cosmological simulations with box sizes of 30 h$^{-1}$ Mpc and
80h$^{-1}$ Mpc respectively.  The dot-dashed blue line is for a
non-cosmological simulation of a ram-pressure stripped galaxy.  In the
top-right panel, there is a line at the critical ratio of 0.06, which
determines if the next finer level ($l+1)$ takes more (above) or less
(below) CPU time than level $l$.}
\label{fig:scaling}
\end{figure}

As, the figure demonstrates, the $f_l$ values all show a sharp decline
with level, dropping as power-laws with the level $l$.  Focusing first
on the ``refine-everywhere" cosmological runs, which both show similar
behavior despite the different box sizes and redshifts at which we
collect the data.  In fact, we find these results are very typical for
cosmological runs, with only slight variations depending on the exact
problem that is being simulated.  The top-right panel shows the ratio
of $f_{l}/f_{l+1}$ which, remarkably, hovers around the critical value
of 0.0625 determined earlier.  This implies that the relative CPU
usage of each level, shown in the lower-left panel, is nearly flat,
and that going to additional levels of refinement only increases the
amount of calculation by a factor of roughly $1/l_{\rm max}$.  Even in
the most extreme case, the L80 run, the last level adds only 25\% to
the computation time.

The memory used by each level, on the other hand, scales as

\begin{equation}
{\rm memory} = C_2 \sum_l f_l N_{\rm root}^3 2^{3l}
\end{equation}

and the terms in this sum are shown in the bottom-right panel of
Figure~\ref{fig:scaling}.  This demonstrates that the memory usage is
dominated by the top three levels, and adding additional levels 
adds minimally to the memory usage of the run.  We have empirically
tested this scaling for cosmological simulations and found it to be
reasonably accurate (although the increase is typically slightly
larger than found here, usually 30-50\%, probably because of
less-than-ideal parallel scaling).

The physical reason for the $f_{l+1}/f_l$ ratios found in the
cosmological simulations appears to be due to a combination of the
density structure of individual clouds and to the distribution of the
clouds themselves.  In particular, we note that for a $\rho \propto
r^{-2}$ density profile, the Lagrangian refinement criteria typically
used in such simulations produces an $f_{l+1}/f_l$ ratio of 1/8 for a
single, resolved cloud.  Of course, at some point we would resolve the
flat density of individual clouds and the ratio would climb, making
further refinement more costly; however, we are not yet in this regime
in this example.

These conclusions are, however, highly dependent on the type of run
being performed.  We contrast this cosmological case to the simulation
of a galaxy formation run, as shown by the dot-dashed curves in
Figure~\ref{fig:scaling}.  Note that grid levels 1 and 2 in this
simulation are statically refined to ensure that the Lagrangian volume
of the galaxy halo is resolved to at least level 2 at all times. More
importantly, the $f_{l+1}/f_{l}$ ratio of the highest levels are
systematically slightly larger than the critical 0.0625 value,
indicating that the CPU time is dominated by the most refined levels,
as shown in the bottom-left panel.  The memory usage is dominated by
levels 2 and 3, the static levels, as the bottom-right panel
demonstrates.  In this case, adding more levels of refinement would
substantially increase the cost of the simulation.

We have focused thus far on scaling when increasing $l_{\rm max}$, but
without changing the refinement criteria.  However, we can also keep
the levels fixed and modify the refinement criteria.  For cosmological
runs, this is typically done by increasing the mass resolution (and
simultaneously decreasing the dark matter particle mass), and adding
additional linear small-scale modes to the initial conditions.  The
effect of this is to boost the $f_l$ values at all levels (except for
the root grid where $f_0$ is already 1) by a factor linearly
proportional to the increase in the mass resolution.  This is because
the refinement criteria typically boosts the number of refined cells
on the first and subsequent levels by this factor.  Again, we have
empirically tested that this scaling is approximately valid, provided
that we keep the maximum level constant.

Putting this together, we find an approximate scaling for the
computational time required for cosmological simulations:

\begin{equation}
t_{\rm SU} \propto M_{\rm res}^{-1} l_{\rm max}.
\end{equation}

This is approximately accurate, and gives users some way to estimate
compute times for \enzo\ calculations.  

\subsection{GPU parallelization and scaling}

Many of today's leading supercomputers use a heterogeneous
computing platform:
on a single node of a distributed-memory platform, a multi-core CPU is
often paired with one or more many-core accelerators.  One programming
model that has been shown to successfully take advantage of these
hardware accelerators is to run the serial component of the algorithm
on the CPU, and the vector-parallel part of the algorithm on the
many-core hardware accelerator. Due to the much higher computational
performance of the many-core accelerators, if a code can be ported to
effectively utilize this heterogeneous architecture, massive speedup
in simulation performance may be achieved.  This heterogeneous trend
is likely to continue in future supercomputers, due to the relatively
low energy needs of hardware accelerators as well as other
factors. Thus, to ensure \enzo's efficiency on future high-end
computational platforms, some of the most time-consuming parts of
\enzo\ have been ported to many-core architecture.  Since GPUs are
currently the most popular many-core accelerator, NVIDIA's CUDA C/C++
programming technology was chosen to port some \enzo\ modules to
NVIDIA GPUs.

Thus far, the PPM and Dedner MHD solvers (described in
Sections~\ref{sec.hydro.ppm} and~\ref{sec.num.hydro-muscl},
respectively), have been ported to take advantage of GPUs using CUDA.
Since the CPU and GPU on a node have access to separate pools of RAM,
fields updated by other modules will be transferred to the GPU before
calling the GPU fluid solver.  Correspondingly, after calling the GPU
solver, the updated fields will be transferred back to CPU. This
ensures that the GPU-parallelized solvers can work correctly with
other parts of the code that have not yet been ported to the GPU, as
well as with the communication infrastructure within \enzo.  In
addition, the GPU solver supports SAMR, which is not a trivial task.
As discussed in Section~\ref{sec.amr}, one of the key steps in \enzo's
AMR implementation is flux correction, which is required when each
level of resolution is allowed to take its own time step.  In the GPU
version of the solvers, the fluxes are calculated on the GPU and only
the fluxes required for flux correction are transferred back to the
CPU.  This reduces data transfer overhead, which can be substantial in
a heterogeneous architecture of this sort.

The key step in porting to many-core architectures such as the GPU is
exposing the massive parallelism inherent in the algorithm. Due to the
explicit, directionally-split stencil pattern of both the PPM and
Dedner MHD solvers, they are inherently massively parallel and thus
should be a good fit for hardware acceleration. Both solvers essentially
contain two parts -- computation of fluxes, and a cell update.
When computing fluxes, the basic procedure is to compute the flux at
each cell-interface given the inputs from neighboring cells. In the
cell update part, the fundamental computation is updating the
cell-centered values using the previously-computed fluxes. In the CPU
serial code, a loop over the grid is used, where the loop body
contains these basic computations. In both parts,
algorithmically-different loop iterations are completely independent
of each other. Thus, the natural parallelization scheme is to map
one GPU thread to one iteration of the loop. However, the original CPU
code, which operates on each grid as a serial process, does not
completely expose this parallelism as some small temporary arrays are
reused among loop iterations.  This re-use of arrays introduces data
dependency among loop iterations, which is undesirable for GPU
parallelization.  Because of this re-use of arrays, the main change in
porting the serial CPU solvers to allow massively parallel GPU
computation was replacing those shared temporary arrays by larger
temporary arrays that are not shared among loop iterations. This
change exposed the massive parallelism in the algorithm, which could
then be accelerated in a straightforward manner using CUDA.

To illustrate the speedup provided by porting two solvers to GPUs, we
show the results of a weak scaling test of driven MHD turbulence on a
uniform mesh in Figure~\ref{fig:gpu_scaling}.  This particular problem
type contains no physics other than the equations of ideal
MHD, and thus is representative of the type of calculation than can currently
benefit from the GPU-optimized solvers in \enzo.  In this scaling
test, we use the Dedner MHD solver, which runs on both CPUs and GPUs
and has been tuned to maximize performance on both platforms.  The
benchmarking platform is a Cray XK6 system with a Gemini
interconnect. Each node has a 16-core AMD Opteron 6272 CPU and a
single NVIDIA Tesla K20 GPU, which has 2,496 CUDA cores and a maximum
theoretical speed of 1.17 Tflops for double-precision computations.  For
the CPU run, all the 16 cores in each node are used by launching 16
MPI processes per node. For the GPU run, 16 MPI processes are also
launched per node, and all processes on this node share the single GPU
on that node. This can work because NVIDIA's MPS (Multi-Process
Service) technology allows multiple processes to concurrently use the
same GPU.  A series of weak scaling tests were run, varying from 1 to
8 nodes (16 to 128 MPI processes), with each node containing a cube of
$256^3$ cells (so, a 4-node computation would have a domain that is
$512 \times 512 \times 256$ cells).  Figure~\ref{fig:gpu_scaling}
displays the scaling results in terms of cell updates per second per
node -- as a result, ideal weak scaling is a horizontal line.  In
general, the simulations using the GPU-accelerated MHD solver perform
roughly 5 times better overall on this system than the equivalent
CPU-only calculation run on the same number of nodes.  We note,
however, that this somewhat mis-represents the speedup obtained by
porting the MHD code to GPUs -- in the GPU simulation case, the 16 CPU
cores on each node are mostly idle, as they are used only for
timestep calculation and boundary condition transmission (which are
both computationally inexpensive compared to the fluid solve).  This
means that, effectively, the GPU simulation is 80 times faster than a
single CPU core, so a user would likely see much greater effective
speedup on a system where the ratio of CPU cores to GPUs is lower.


\begin{figure}
\centerline{\includegraphics[width=0.5\textwidth]{figures/gpu_scaling.eps}}
\caption{Weak scaling performance of MHD turbulence using the Dedner
MHD algorithm, which has both CPU and GPU solvers.  Each node of the
machine (a Cray XK6) contains one 16-core AMD Opteron 6272 CPU and 1
NVIDIA Tesla K20 GPU.  In this test, there are $256^3$ cells/node.
The y-axis shows cell updates per second per node, which for ideal
weak scaling is a horizontal line.  Blue line: CPU solver.  Red line:
GPU solver.}
\label{fig:gpu_scaling}
\end{figure}


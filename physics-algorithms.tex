\section{Physics and Algorithms}
\subsection{Physical Equations}
\label{sec.physical.eq}
The spatial and temporal evolution of a compressible fluid is governed by the
Euler equations. The Euler equations are a set of hyperbolic conservation laws
governing the density $\rho$, velocity $\mathbf{v}$, and total specific energy
$E$ (specific kinetic energy $\frac{1}{2}\rho \mathbf{v}^2$ plus specific internal
energy $u$). These quantities $\mathbf{U}$ and their respective fluxes
$\mathbf{F}(\mathbf{U})$, defined through the Euler equations, are conveniently
expressed in vector form as
%
\begin{equation}
    \mathbf{U} =
    \left(
    \begin{array}{c}
        \rho \\
        \mathbf{v} \\
        \rho e \\
    \end{array} \right),
    \quad
    \mathbf{F}(\mathbf{U}) =
    \left(
    \begin{array}{c}
        \rho\mathbf{v} \\
        \rho\mathbf{v}\mathbf{v}^T + P \\
        \rho e\mathbf{v} + P\mathbf{v} \\
    \end{array}
    \right),
\end{equation}
%
where P is the pressure. In this notation the Euler equations can be expressed
in the form
%
\begin{equation}
    \label{eq.euler}
    \frac{\partial \mathbf{U}}{\partial t} + \div \mathbf{F} = 0.
\end{equation}
%
As stated the system of equations are not closed; there are more variables then
equations. Thus, a constraint is needed and is supplied by the equation of state, which is typically given by the ideal gas relation
%
\begin{equation}
    P = \rho(\gamma - 1)u,
\end{equation}
%
where $\gamma$ is the ratio of specific heats.

These equations can be solved by the finite-volume approach, which is a
discretization of the domain into finite sized disjoint cells and evolves their
spatial averaged $\mathbf{U}$ values. Specifically, applying equation
(\ref{eq.euler}) to every cell $i$ with volume $V_i$ and performing Gauss'
theorem to convert the volume integral to a surface integral results in
%
\begin{equation}
    \label{eq.euler.int}
    \frac{d\mathbf{Q}_i}{dt} =
    -\sum_{j}\int\mathbf{F}_{ij}(\mathbf{U})\cdot\mathbf{A}_{ij},
\end{equation}
%
where $\mathbf{A}_{ij}$ is the cell's surface area normal and $\mathbf{Q}_i$
is the volume integral of $\mathbf{U}_i$,
%
\begin{equation}
	\mathbf{Q}_i =
    \left(
    \begin{array}{c}
    	m_i \\
        \mathbf{p}_i \\
        E_i
     \end{array}
     \right) = \int_{V_i}\mathbf{U}_i dV,
\end{equation}
where $m_i$, $\mathbf{p}_i$ and $E_i$ are cell's total mass, momentum, and
energy, respectively. Here, an assumption is taken that the cell's volume
is a polyhedron such that the surface integral can become a sum over all
polygon faces. Additionally, time is discretized leading to a finite difference
update of the cell's quantities given by
%
\begin{equation}
    \label{eq.flux.update}
    \mathbf{Q}_i^{n+1} = \mathbf{Q}_i^n - \Delta t\sum_j
    \mathbf{\hat{F}}_{ij}^{n+1/2} A_{ij}.
\end{equation}
%
In this expression, the flux $\mathbf{\hat{F}}^{n+1/2}$ is taken to be a time
average and is constant across the cell face. To be able to use equation
\ref{eq.flux.update} we must make estimates of
$\mathbf{\hat{F}}^{n+1/2}$ and $A_{ij}$ to the proper order of accuracy. Sections \ref{sec.riemann} and
\ref{sec.tessellation}, respectively, will go into detail on how these quantities are
estimated.

\subsection{Tessellation}
\label{sec.tessellation}
As stated in Section \ref{sec.physical.eq}, the Euler equations, as written in
(\ref{eq.flux.update}), assume that the cells are in the form of polyhedrons.
There are many ways of partitioning space that meet this criteria but for this
paper we will restrict our attention to the Voronoi tessellation. The Voronoi
tessellation partitions the space into a set of disjoint polyhedrons for a
given set of points. The set of points are called mesh generators and for each
generator there is a corresponding region consisting of all points that are closer
to that generator than any other. Thus, for each mesh generating pair there
lies a polygon that is equidistant. This geometric property will be exploited
in several ways. For the rest of this paper we will refer to the polyhedra
associated with the mesh generator as a cell and the polygons making up the
polyhedra as faces.

The creation of the Voronoi tessellation can be performed by first constructing
the Delaunay triangulation. The Delaunay tessellation is the dual graph of the
Voronoi tessellation and is defined as the triangulation of points such that
no point in the set lies inside the circumcircle of any of the triangles, in
2d. In 3d the triangles become tetrahedron but the same definition applies.
Remaining in 2d for simplicity, the points are the vertices of the triangles.
The dual aspect refers to the fact that each triangle edge corresponds to a
Voronoi face and each Voronoi vertex corresponds to the center of the
circumcirlce of the triangle. 

\subsection{Fluid Update}
\subsubsection{Reconstruction}
% Need some introductory text
To solve the Riemann problem, primitive values are needed at the face. As a first
choice, cell center values can be used. Although for more accurate results a 
method is needed to extrapolate the cell center values to the center of mass of 
the face. At first order, in space and time, can be constructed by Taylor-series 
expanding the primitive values
%
\begin{equation}
	\label{eq.taylor}
	\mathbf{W}' = \mathbf{W} + \frac{\partial\mathbf{W}}{\partial\mathbf{r}}
    	(\mathbf{f}-\mathbf{s}_i) + \frac{\partial\mathbf{W}}
        {\partial t}{\Delta t}.
\end{equation}
%
The expansion has two unknowns, the spatial derivative
$\frac{\partial\mathbf{W}}{\partial \mathbf{r}}$ and the time derivative
$\frac{\partial\mathbf{W}}{\partial t}$. However, both derivatives
are related through the Euler equations in primitive form
%
\begin{equation}
    \frac{\partial\mathbf{W}}{\partial t}  + \mathbf{A}
    	(\mathbf{W})\frac{\partial\mathbf{W}}{\partial\mathbf{r}} = 0,
    \quad
    \mathbf{A}(\mathbf{W}) =
    \left(
    \begin{array}{ccc}
        \mathbf{v} & \rho & 0 \\
        0 & \mathbf{v} & 1/\rho \\
        0 & \gamma P & \mathbf{v}
    \end{array}
    \right).
\end{equation}
%
Thus, only the spatial derivatives need to be calculated. For the
calculation we follow the method presented in (Springel), which is
summarized below. Given a cell $i$ the components of 
$\frac{\partial\mathbf{W}}{\partial \mathbf{r}}$, which are gradients
of the primitive variables, are given by
%
\begin{equation}
	\nabla\phi_i = \frac{1}{V_i}\sum_{j}A_{ij}
    	\left(\left[\phi_i - \phi_j\right]\right) -
        \frac{\phi_i + \phi_j}{2}
        \frac{\mathbf{r}_{ij}}{r_{ij}}.
\end{equation}
%
The sum is over all neighbors $j$ of particle $i$, $\phi_i$ and
$\phi_j$ are the scalar field values at each cell center respectively,
$\mathbf{r}_{ij} = \mathbf{r}_i - \mathbf{r}_j$ is the separation vector
with magnitude $r_{ij} =|\mathbf{r}_{ij}|$. As constructed, the gradients
are second order in space for smooth flows. However, in the presence of
shocks, numerical instabilities may arise and therefore the reconstruction
must be reduced. To deal with this a slope limiter is used. We employ two
limiters. The first one is the method used by AREPO and begins with
the calculation of
%
\begin{equation}
		\psi_{ij} = \left\{
  		\begin{array}{@{}ll@{}}
    		(\phi_i^{max} - \phi_i)/\Delta \phi &
            	\text{for}\ \Delta\phi_{ij} > 0 \\
            (\phi_i^{min} - \phi_i)/\Delta \phi &
            	\text{for}\ \Delta\phi_{ij} < 0 \\
            1 & \text{for}\ \Delta\phi_{ij} = 0,
  		\end{array}\right.
\end{equation}
%
where $\phi_i^{max} =$ max$(\phi_j)$ and $\phi_i^{min} =$ min$(\phi_j)$ are
the maximum/minimum values across all neighbors of $i$ and
$\Delta\phi_{ij} = \nabla \phi_i \cdot (\mathbf{f}_{ij} - \mathbf{s}_i)$. Then 
the minimum of all $\psi_{ij}$s, associated with each primitive field, is found 
producing a single scalar value
%
\begin{equation}
	\alpha_i = min(1, \psi_{ij})
\end{equation}
%
 that is used to limit the gradient
%
\begin{equation}
	\nabla \phi_i' = \alpha_i \nabla \phi_i
\end{equation}

\subsubsection{Riemann Solve}
\label{sec.riemann}
The Riemann problem consist of two constant states separated by an interface.
The solution to this problem is the formation of three waves emanating from the
interface. These waves are associated with the eigenvalues of the Euler equations
and each carry a jump in the characteristic variables. The goal of the Riemann
solver is to estimate these nonlinear waves and construct the fluxes at the
interface. Four purposes, we have elected to use the HLL, HLLC, and Exact Riemann
solvers. 
 
The HLL solver ignores the contact in solving the Riemann problem. In this case
only two waves are considered.

\subsection{Grid Motion}
As currently constructed the method outlined in solving the Euler equations are 
for static meshes only. In our case we allow the mesh to move, meaning the mesh
generators are given some velocity $\mathbf{w}_i$. Thus, equation \ref{eq.euler}
must be augmented to account for an advection term produced by the movement
of the face. The updated Euler equations become
%
\begin{equation}
	\label{eq.euler.moving}
    \mathbf{F}_m(\mathbf{U}) = \mathbf{F}_s(\mathbf{U})
    	- \mathbf{U}\mathbf{w}^T =
    \left(
    \begin{array}{c}
        \rho\mathbf{v} \\
        \rho\mathbf{v}\mathbf{v}^T + P \\
        \rho e\mathbf{v} + P\mathbf{v} \\
    \end{array}
    \right) -
    \left(
    \begin{array}{c}
        \rho\mathbf{w} \\
        \rho\mathbf{v}\mathbf{w}^T \\
        \rho e\mathbf{w} \\
    \end{array}
    \right).
\end{equation}
%
In practice equation  is not used because of its unstable
numerical behavior (Parkmor). The equation can become numerical stable by
solving the fluxes in the rest frame of the face. At the face the conserved 
variables and fluxes become
%
\begin{equation}
	\mathbf{U}' =
    \left(
    \begin{array}{c}
        \rho \\
        \rho(\mathbf{v-w}) \\
        \rho e' \\
    \end{array} \right),
    \quad
   	\mathbf{F}'(\mathbf{U}') = 
    \left(
    \begin{array}{c}
        \rho(\mathbf{v-w}) \\
        \rho(\mathbf{v-w})(\mathbf{v-w})^T + P \\
        \rho e'(\mathbf{v-w}) + P(\mathbf{v-w}) \\
    \end{array}\right) =
    \left(
        \begin{array}{c}
        F_\rho' \\
        F_\mathbf{v}' \\
        F_e' \\
    \end{array}
    \right).
\end{equation}
%
Notice $\rho$ and $P$ remain unchanged however the $e$ transform to
$e' = e -\frac{1}{2}\mathbf{v}^2 + \frac{1}{2}(\mathbf{v-w})^2$. Now
the fluxes need to be transformed back to the lab frame
%
\begin{equation}
	\label{eq.euler.face}
    \mathbf{F}_m(\mathbf{U}) = \mathbf{F}_s(\mathbf{U})
    	- \mathbf{U}\mathbf{w}^T = \mathbf{F}'(\mathbf{U}') +
    \left(
    \begin{array}{c}
        0 \\
        \rho\mathbf{w}(\mathbf{v-w})^T \\
        \rho(\mathbf{vw})(\mathbf{v-w}) -\frac{\rho}{2}
        	\mathbf{w}^2(\mathbf{v-w}) + p\mathbf{w} \\
    \end{array}
    \right),
\end{equation}
%
where the terms have been picked to make equation \ref{eq.euler.face} consistent with
equation \ref{eq.euler.moving}. This equation can be restated in terms of the rest
frame fluxes,
%
\begin{equation}
	\label{eq.final.euler}
    \mathbf{F}_m(\mathbf{U}) = \mathbf{F}_s(\mathbf{U})
    	- \mathbf{U}\mathbf{w}^T = 
    \left(
    \begin{array}{c}
        F_\rho' \\
        F_\mathbf{v}' + \mathbf{w} F_\mathbf{v}'^T \\
        \mathbf{w}F_\mathbf{v}' + \frac{1}{2}F_\rho' \mathbf{w}^2
    \end{array}
    \right).
\end{equation}
%
Thus, after solving the riemann problem in the rest frame of the face the fluxes
can be easily transformed back into the lab frame. Finally, to make use of equation
\ref{eq.final.euler} with \ref{eq.taylor} the primitive variables need to transformed
by the following
%
\begin{equation}
    \mathbf{W}' = \mathbf{W} - 
    \left(
    \begin{array}{c}
        0 \\
        \mathbf{w} \\
        0 \\
    \end{array}
    \right).
\end{equation}
%
This ensures that the correct values are used in the riemann solver.

\subsubsection{Time Integration}
The time integration using equation \ref{eq.euler.int} with \ref{eq.taylor} is
a form of the MUSCL-Hancock scheme. For static meshes, the scheme is second
order accurate in space and time. However, letting the mesh move introduces
inaccuracies due to ignoring the effect of the mesh deformation during a time 
step $\Delta t$. This can be corrected by adopting a Runge-Kutta type scheme, that
uses information from the beginning and the end of a time step instead of mid
point estimations. Specifically, we employ the method outlined by (Pakmor)
which updates the conservative variables by the following
\begin{equation}
	\begin{array}{rcl}
		\mathbf{W}_i' & = & \mathbf{W}_i^n + 
        	\Delta t\frac{\partial\mathbf{W}}{\partial t}, \\
        \mathbf{r}' & = & \mathbf{r}^n + \Delta t\mathbf{w}^n, \\
        \mathbf{Q}_i^{n+1} & = & \mathbf{Q}_i^n -
        	\frac{\Delta t}{2}\left(\sum_j A_{ij}^n\mathbf{\hat{F}}_{ij}^n
            (\mathbf{W}^n) + \sum_j A_{ij}'\mathbf{\hat{F}}_{ij}'
            (\mathbf{W}')\right), \\
        \mathbf{r}^{n+1} & = & \mathbf{r}'.
    \end{array}
\end{equation}
Here we are taking an average of the fluxes from the beginning and the end
of the time step. The flux at the beginning of the time step is constructed
with the current state of the mesh with the primitive values extrapolated to
the face. Then the mesh generators move to their final position
and the mesh is reconstructed. A new flux is constructed with new geometric
quantities, however, the primitive values have been extrapolated in time from
the beginning of the time step. At first glance, it seems that the mesh has
to be constructed twice per time step. However, the generator velocity is
treated constant throughout the time step resulting the final mesh to be equal to
the beginning mesh of the next time step. Thus, the mesh need only to
be constructed once per time step while the fluxes have to be calculated
twice per time step. This method is not truly a Runge-Kutta scheme because
of the time extrapolation but more a mixture of Runge-Kutta and MUSCL-Hancook
scheme that has been shown to be second order accurate in space and time.

\subsubsection{Regularization}
Allowing the mesh generators to move with the local fluid velocity 
$\mathbf{w}_i$ can lead to cells that are elongated or mesh generators close to 
given face. This results, to an unstable evolution of the cells because their 
faces can move rapidly relative to the generator velocity (Duffell). To 
counteract this issue (Springel) proposed a correction term that would steer the 
generator towards its center of mass. This effectively causing the cell to 
become rounder, thus mitigating the issue. The correction term is defined as
%
\begin{equation}
	\mathbf{w}_i' = \mathbf{w}_i + \chi\left\{
  		\begin{array}{@{}ll@{}}
    		0, & \text{for}\ d_i/(\eta R_i) < 0.9 \\
    		c_i\frac{\mathbf{s}_i - \mathbf{r}_i}{d_i}
            	\frac{d_i-0.9\eta R_i}{0.2\eta R_i}, 
            	& \text{for}\ 0.9 \leq d_i/(\eta R_i) < 1.1 \\
            c_i\frac{\mathbf{s}_i - \mathbf{r}_i}{d_i},
               	& \text{for}\ 1.1 \leq (d_i)/(\eta R_i).
  		\end{array}\right.
\end{equation}
%
Here $R_i$ is the effective radius of the cell, $(V_i/\pi)^{1/2}$ for 2d and
$(3V_i/4\pi)^{1/3}$ for 3d, $d_i$ is the distance between the generators
position $\mathbf{r}_i$ and center of mass $\mathbf{s}_i$, $c_i$ is the local sound speed. $\chi$ and $\eta$ are tunning parameters which are typically set 
to 0.25 and 1.0, respectively.

\subsection{External Boundary Conditions}
For each simulation a boundary condition must be defined. At this moment, only
reflection and periodic boundaries are implemented. Our domains are restricted to
rectangular domains with arbitrary aspect ratios. The implementation of both
boundaries make use of ghost particles. These ghost particles are created during
the mesh construction and carry all particle information to participate in the 
integration step. From the simulation perspective, they are treated as real particles,
however, after a time step all ghost particles are discarded and new ghost particles
are formed with the relevant updated particle information.

\subsubsection{Periodic}
Periodic boundaries are formed by examining the circumcirlce of each real particle
in the tessellation. If this value intersects the domain boundary the particle is
flagged for ghost construction. The flagged particle is then shifted periodically in
all dimensions including corner cases and tested for boundary intersection. If 
intersection occurs a ghost particle is formed from the particle but with the
appropriate position data.

\subsubsection{Reflecting}
The reflection boundary parallels the periodic case with exception that particles
are not periodically shifted. Instead, the flagged particle is mirrored across the
minimum and maximum of each boundary dimension. If intersection occurs, again, a ghost
particle is formed, however, the sign of the normal velocity component is flipped.
This ensures that the mass flux vanishes on the surface of the boundary.

\subsection{Gravity}
In the presence of gravity the Euler equations \ref{eq.euler} are modified by
a source term
%
\begin{equation}
    \frac{\partial \mathbf{U}}{\partial t} + \div \mathbf{F} =\left( 
    \begin{array}{c}
    	0 \\
        -\rho\nabla\mathbf{\Phi} \\
        -\rho\mathbf{v}\nabla\mathbf{\Phi}
    \end{array}\right).
\end{equation}
%
Note that the gravitational potential $\mathbf{\Phi}$ only affects the
momentum and energy. The source of the potential can be prescribed by
an external source or by the self gravity of the gas. In the latter case
the potential is given by Poisson's equation
%
\begin{equation}
	\nabla^2\mathbf{\Phi} = 4\pi G\rho
\end{equation}
For the moment assume that $\mathbf{\Phi}$ is given. Then the equations
\ref{eq.taylor} and \ref{eq.euler.int} can be easily supplemented to
include the gravitational force. First, the time derivatives in the
reconstruction equation are replaced by
%
\begin{equation}
    \frac{\partial\mathbf{W}}{\partial t}  + \mathbf{A}
    	\left(\mathbf{W}\right)\frac{\partial\mathbf{W}}{\partial\mathbf{r}}
        = \left(
        	\begin{array}{c}
            0 \\
            -\nabla\mathbf{\Phi} \\
            0
            \end{array}
         \right),
\end{equation}
%
In this case the time extrapolated variables include a gravitational
component. Second, the momentum and energy is updated during the flux update
%
\begin{equation}
	\begin{array}{rcl}
        \Delta\mathbf{p}_i & = &
        	\frac{\Delta t}{2}\left(\sum_j A_{ij}^n\mathbf{\hat{F}}_{ij,\mathbf{p}}^n
            (\mathbf{W}^n) + \sum_j A_{ij}'\mathbf{\hat{F}}_{ij,\mathbf{p}}'
            (\mathbf{W}')\right) - \frac{1}{2}\Delta t\left( 
        	m_i\nabla_i\mathbf{\Phi} + m_i'\nabla_i\mathbf{\Phi}'\right)\\
        \Delta E_i & = &
        	\frac{\Delta t}{2}\left(\sum_j A_{ij}^n\mathbf{\hat{F}}_{ij,E}^n
            (\mathbf{W}^n) + \sum_j A_{ij}'\mathbf{\hat{F}}_{ij,E}'
            (\mathbf{W}')\right) - \frac{1}{2}\Delta t\left( 
        	m_i\mathbf{v}_i\nabla_i\mathbf{\Phi} +
            m_i'\mathbf{v}_i'\nabla_i\mathbf{\Phi}'\right)\\
    \end{array}
\end{equation}
%
\subsubsection{Constant}
\subsubsection{Tree}

\subsection{Chemistry and Cooling}
\subsubsection{Grackle}

\subsection{Additional Physics}
\subsubsection{Conduction}
\subsubsection{Magnetic Fields}

\subsection{Parallelism}
\subsubsection{Partitioning}
For domain decomposition we follow the approach of (Springel) and use
a space filling curve. Here every particle position is mapped onto a
1D line. The line is decomposed into a number or roughly equal pieces.
The number of segments is equal to the number of processor for the
give run.

In practice, for each particle a Peano-Hilbert key is computed. The key

\subsubsection{Ghost particle copying}

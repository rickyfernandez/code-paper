\section{Physics and Algorithms}
\subsection{Physical Equations}
The spatial and temporal evolution of a compressible fluid is governed by the Euler equations.
The Euler equations are a set of hyperbolic conservation laws governing the density $\rho$,
velocity $\mathbf{v}$, and total specific energy $E$ (specific kinetic energy
$\frac{1}{2}\rho \mathbf{v}^2$ plus specific internal energy $u$). These quantities $\mathbf{U}$
and their respected fluxes $\mathbf{F}(\mathbf{U})$, defined through the Euler equations, are
conviently expressed in vector form as
%
\begin{equation}
    \mathbf{U} =
    \left(
    \begin{array}{c}
        \rho \\
        \mathbf{v} \\
        \rho e \\
    \end{array} \right),
    \quad
    \mathbf{F}(\mathbf{U}) =
    \left(
    \begin{array}{c}
        \rho\mathbf{v} \\
        \rho\mathbf{v}\mathbf{v}^T + P \\
        \rho e\mathbf{v} + P\mathbf{v} \\
    \end{array}
    \right),
\end{equation}
%
where P is the pressure. In this notation the Euler equations can be expressed in the form
%
\begin{equation}
    \label{eq.euler}
    \frac{\partial \mathbf{U}}{\partial t} + \div \mathbf{F} = 0.
\end{equation}
%
As currently stated the system of equations are not closed; more variables then equations. Thus,
a constraint is needed and is supplied by the equation of state
%
\begin{equation}
    P = \rho(\gamma - 1)u,
\end{equation}
%
where $\gamma$ is the ratio of specific heats.

These equations can be solved by the finite-volume approach, which is a discretization
of the domain into finite sized disjoint cells and evolves their spatial averaged
$\mathbf{U}$ values. Specifically, applying eq. \ref{eq.euler} to every cell $i$ with
volume $V_i$ and performing Gauss' theorem to convert the volume integral to a
surface integral results in
%
\begin{equation}
    \label{eq.euler_int}
    \frac{\partial{\mathbf{Q}}_i}{\partial{t}} =
    -\sum_{j}\int\mathbf{F}_{ij}(\mathbf{U})\cdot\mathbf{A}_{ij},
\end{equation}
%
where $\mathbf{Q}_i$ is the volume integral of $\mathbf{U}_i$ and $\mathbf{A}_{ij}$ is the
cells surface normal. Here, an assumption was taken that the cells volume is a polyhedron
such that the surface integral can become a sum over all polygon faces. Furthermore, time
is discretize leading to an update of the cells quanitites
%
\begin{equation}
    \mathbf{Q}_i^{n+1} = \mathbf{Q}_i^n - \Delta t\sum_j \mathbf{\hat{F}}_{ij}^{n+1/2} A_{ij}.
\end{equation}
%
In this expression the flux $\mathbf{\hat{F}}^{n+1/2}$ is taken to be a time averaged and
constant across the cell face.

\subsection{Tessellation}

\subsection{Fluid Update}
\subsubsection{Reconstruction}
\subsubsection{Riemann Solve}
\subsubsection{Flux Computation}
\subsubsection{Time Integration}

\subsection{Grid Motion}
\subsubsection{Regularization}

\subsection{External Boundary Conditions}
\subsubsection{Reflecting}
\subsubsection{Periodic}

\subsection{Gravity}
\subsubsection{Constant}
\subsubsection{Tree}

\subsection{Chemistry and Cooling}
\subsubsection{Grackle}

\subsection{Additional Physics}
\subsubsection{Conduction}
\subsubsection{Magnetic Fields}

\subsection{Parallelism}
\subsubsection{Partitioning}
\subsubsection{Ghost particle copying}

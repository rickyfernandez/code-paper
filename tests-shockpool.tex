\subsubsection{Shock pool}
\label{sec.tests.shockpool}

The next problem is similar to the last one but instead examines how a
shock with a Mach number of 2 passes in and out of a static refined
region (again defined from $z=0.25$ to 0.75).  The density and
pressure in the domain are initially set equal to 1.0 with zero
velocity.  At $t=0$ the left boundary is set with the density,
pressure and velocity appropriate for a $\mathcal{M}=2$ shock wave.

In Figure~\ref{fig.shockpool}, we show the evolution of the shock wave
at three times, again corresponding to just before entering the refine
region (left column), after entering the refined region (center), and
after exiting the refined region (right).  We examine the same set of
three solvers as in the previous test problem.

Beginning with PPM (top row of Figure \ref{fig.shockpool}), we see
that this method captures the shock in a small number of zones with
only a small amount of oscillation.  Upon entering the refined region,
the shock finds itself broader than the natural width of the scheme
(since the cell spacing is decreased by a factor of 2), and so the
shock front contracts, causing a slight entropy perturbation in the
post-shock gas.

For \zeus\ (middle row of Figure~\ref{fig.shockpool}), the shock is
broadened because of artificial viscosity and there are slightly more
post-shock oscillations, although again quite mild.  The impact of
entering and exiting the refined region is somewhat larger than in
PPM; however, the most noticeable difference is the incorrect position
of the shock front. This is due to the fact that the scheme is not
energy-conserving (see also the Sedov problem in
Section~\ref{sec.tests.sedov}), and has very little to do with the
passage through the refined region.

Finally, the MUSCL scheme produces a shock that is intermediate in
width between the two previous cases.  This method is
energy-conserving, and thus correctly reproduces the shock speed.  The
oscillations are mild except for the cell immediately outside of the
refined region upon exiting.

\begin{figure}
\begin{center}
\includegraphics[width=0.8\textwidth]{figures/ShockPool}
\caption{This plot shows, for each column, three snapshots at $t=0.1,
0.15$ and 0.35 of a $\mathcal{M}=2$ shock as it propagates through the
grid with a static refined region extends from $x = 0.25$ to 0.75
(shown in grey in each panel).  Each row shows the result for a
different solver (PPM/\zeus/MUSCL from top to bottom).  The solid line
shows the analytic solution.  Note that we focus each panel on a small
region of the entire domain to better show the shock front.}
\label{fig.shockpool}
\end{center}
\end{figure}

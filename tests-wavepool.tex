\subsubsection{Wave pool}
\label{sec.tests.wavepool}

In this one-dimensional test we pass a short wavelength linear wave
through a static singly-refined region.  The full domain is from 0 to
1 and the refined region is set from 0.25 to 0.75 at all times.  We
modify the left boundary consistent with a single linear sound wave
with density given by $\rho(x,t) = \rho_0 (1 + A \sin(kx - \omega
t))$, where the amplitude is $A = 0.01$ and the wavelength $\lambda =
k/2\pi = 0.1$.  Similar expressions exist for the pressure and
velocity.  The initial, unperturbed density and pressure are set to
unity and we adopt $\gamma = 1.4$.  The (unrefined) root grid is
covered with 100 cells, resulting in a wavelength for the linear wave
of only 10 cells.  This is, therefore, a challenging problem for hydro
methods.  We are particularly interested in any reflection or
artifacts introduced by the wave entering and exiting the refined
regions.

Figure~\ref{fig.wavepool} shows the evolution of the wave at three
different times ($t = 0.2, 0.3$ and 0.8) for three of our solvers
(PPM, \zeus, and MUSCL).  In all panels, the shaded regions denote the
statically refined region. The leftmost column shows the wave just
before it enters the refined region so that we can gauge how the
solver is operating in the absence of AMR.  The center column shows
the wave after it has fully entered the refined region.  The rightmost
column shows the wave well after it has exited the refined region.

With the PPM solver, we notice that even before the wave reaches the
refined region it has been slightly damped. This is not unexpected for
such a short wavelength mode, even for higher-order solvers such as
PPM.  The remaining panels demonstrate that the wave cleanly enters
and exits the refined region.  No significant reflection is seen on
either entry or exit, and the amount of damping is mild.

For the \zeus\ solver, we see that even before it has entered the
refined region there are small oscillations excited behind the wave,
although it is also worth noting that the wave itself is beautifully
propagated without significant damping or phase errors.  In this test
we use only our standard, low amount of quadratic artificial
viscosity.  The trailing oscillations could be damped by additional
viscosity, but we do not add any in order to be sensitive to any
artifacts at the grid boundary.  The remaining panels show that the
trailing oscillations continue, but do not generate any additional
noise. We note that the end result is similar to the case without any
refined region.

Finally, the MUSCL solver also shows a very clean entry and exit from
the refined region without any oscillations, although with mild
damping on exit. However, because we use a piecewise linear
reconstruction, the wave is spread more than with the other methods.

\begin{figure}
\begin{center}
\includegraphics[width=0.8\textwidth]{figures/WavePool.eps}
\caption{This plot shows, for each column, three snapshots at $t=0.2,
0.3$ and 0.8 of a linear wave as it propagates through the domain.  A
static refined region extends from $x = 0.25$ to 0.75 and is shown in
grey in each panel.  The individual cells are also color-coded by
level: blue indicates the root grid, and red is for the refined
region.  Each row shows the result for a different hydrodynamic solver
(top is PPM, middle is \zeus, and the bottom is MUSCL).  The solid
line shows the analytic solution for a linear, undamped wave.  Note
that we focus each panel on a small region of the entire domain to
better show the wave itself.}
\label{fig.wavepool}
\end{center}
\end{figure}

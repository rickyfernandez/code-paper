\section{Introduction}
\label{sec.intro}

The use of numerical simulations for modeling astrophysical compressible 
gas have steadily grown through the rise of computing power and decrease
in price of hardware. Astrophysical problems poses its unique set of challenges
manifesting throught its high spatial and temporal dynamic ranges. Progress
has been made over the years by the development of algorithms that leverage
computing resources to the problem at hand. The two most notable numerical methods 
are Langrangian and Eulerian schemes. A popular example of Langrangian scheme
is the smoothed particle hydrodyanmics developed by ... 

In this scheme
talk  about sph strengths and weakness

talk about eulerian

talk

For example adaptive mesh refinement
allows for dynamic addition of computing grids. This feature allows computing
resources to adaptively  

\subsection{Motivation for ALE}
\label{sec.motivation_ale}

Due to the high spatial and temporal dynamical ranges involved,

\subsection{Motivation for Python}
\label{sec.motivation_python}

Due to the high spatial and temporal dynamical ranges involved,

\subsection{Design Philosophy}
\label{sec.design_philosophy}

Due to the high spatial and temporal dynamical ranges involved,

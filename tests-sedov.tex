\subsubsection{Sedov Explosion}
\label{sec.tests.sedov}

\begin{figure}
\begin{center}
\includegraphics[width=0.4\textwidth]{figures/sedov-ppm-slice.eps}
\includegraphics[width=0.4\textwidth]{figures/sedov-zeus-slice.eps}
\caption{Density slices from the Sedov Blast test at $t = 0.07$. Left:
results using the Piecewise Parabolic Method hydro scheme.  Right:
results using the \zeus\ hydro method. Notably, the \zeus\ shock front
has progressed less far than in the PPM run. This is due to energy
loss when conserving only internal, and not total, energy.}
\label{fig.sedov1}
\end{center}
\end{figure}


\begin{figure}
\begin{center}
\includegraphics[width=\textwidth]{figures/sedov-profiles.eps}
\caption{Radial profiles for the Sedov Blast test at $t =
0.07$. Clockwise from top-left shows density, velocity, internal
energy and pressure.  The black solid line shows the analytic
solution.  The blue dashed line shows the simulation using the PPM
method, and the red dot-dashed line using \zeus.  The \zeus\ result
substantially lags the true result due to total energy not being
explicitly converged.}
\label{fig.sedov2}
\end{center}
\end{figure}

The Sedov Blast test \citep{Sedov1959} models an intense explosion,
initiated by depositing thermal energy into a homogenous distribution
of gas. The result is a strong spherical shock wave centered on the
point of energy injection.  This problem is a popular test of
astrophysical simulation codes for three reasons.  First, this problem
is representative of the astrophysical phenomenon of a supernova
explosion.  Second, an analytical solution to the evolution of the
Sedov blast wave exists, allowing for a direct test of the accuracy of
the code.  The radius of the shock front as a function of time is
given by


\begin{equation} r(t) =
\left(\frac{E_0}{\alpha\rho_0}\right)^{1/5}t^{2/5}
\end{equation}

\noindent where $E_0$ is the initial energy injection, $\rho_0$ is the
background density and $\alpha = 1.0$ for cylindrical symmetry and an
ideal gas with $\gamma = 1.4$; for the full derivation see
\citet{Sedov1959}.  Third, as the spherical shock expands, its
symmetry, or lack thereof, serves to highlight any directional
preferences of the hydrodynamics solvers.  The test presented here is
the two-dimensional version that is included in the \enzo\
distribution. The three-dimensional results from this test, both for
\enzo\ and three other leading astrophysics codes, can be found in
\citet{Tasker2008}.

In the initial state, the box contains a homogenous distribution of
gas at a density of 1 (note that, in the absence of gravity or
radiative cooling, this problem is scale-free and thus without
units). Thermal energy is deposited into a single cell at the center
of the box with $E_0 = 10.0$. The problem is run in two dimensions
with reflecting boundary conditions and a box having a length of $1$
in both directions.  We selecte a top grid of $100 \times 100$ cells
and a maximum of four levels of adaptive refinement, refining by
factors of two on shocks and the slopes of the density and total
energy fields. The exception to this scheme is in the initial
conditions, where additional refinement is placed directly around the
injection point to better resolve spherical geometry. The results are
assessed at $t = 0.07$, which corresponds to a time just before the
shock reaches the box edge (see Figure~\ref{fig.sedov1}).

Figure~\ref{fig.sedov2} shows the radial profiles for the simulation
run with the PPM hydro-solver (blue dashed line) and the \zeus\
hydro-solver (red dash-dot-dot line) together with the analytical
solution (black solid line). Clockwise from the top-left are density,
velocity, internal energy and pressure. PPM matches the analytical
solution extremely well for all quantities. However, the shock front
in the \zeus\ simulation lags behind the analytical position. This can
also be seen in the slices shown in Figure~\ref{fig.sedov1}. The cause
of this discrepancy is that \zeus\ shows a substantial energy loss
during the first few timesteps and produces a diamond-shaped, rather
than spherical, shockfront during this time. After this, the code
correctly conserves energy but this intial energy loss remains clearly
visible in the position of the shock at $t = 0.07$. This problem was
addressed directly by \citet{Clarke2010}, who attributed the source of
the issue to this version of \zeus\ solving the internal, rather than
total, energy equation. In situations with strong energy gradients,
this choice caused an energy loss and the artificial viscosity
produces the direction-dependent shockfront shape. In their paper,
\citet{Clarke2010} present results from an alternative version of
\zeus\ that conserves total energy. This problem is less marked for
smaller energy gradients and it should be noted that the \zeus\ hydro
algorithm's stability and speed make it a highly competitive choice,
despite the disagreements in this test.

\subsubsection{Sod Shock Tube}
\label{sec.tests.sodshock}

We begin the paper test suite with the classic one-dimensional Sod
shock tube problem \citep{Sod78}, which provides a good test of a
hydrodynamical solver's ability to resolve a clean Riemann problem
with clear separation between the three resultant waves.  These waves
consist of a rarefaction fan, a contact discontinuity, and a moderate
shock.  The initial state is $\rho_{\rm L}, P_{\rm L} = 1.0, 1.0$ on
the left of the boundary at $x=0.5$ and $\rho_{\rm R}, P_{\rm R} =
0.125, 0.1$ on the right.  All velocities are initially zero.  In
Figure~\ref{fig.sodshocktube}, we show the density solution at the
final time ($t=0.25$) for three of our hydro solvers -- the spatially
third-order PPM, as well as the two second-order \zeus\ and MUSCL
schemes.  We use 100 cells across the domain, which is a relatively
standard choice in code method papers, and show solutions both with
and without adaptive mesh refinement.  Without AMR (top row in Figure
\ref{fig.sodshocktube}), it is clear that the PPM scheme produces by
far the cleanest solution with all wave families crisply reproduced
(in particular, the contact discontinuity and shock).  \zeus\ and
MUSCL produce similar results, with MUSCL doing a slightly better job
on the rarefaction fan.  In Figure \ref{fig.sodshocktube}, we also
show the integrated absolute deviation from the exact solution,
$||E_1|| = \sum_i \Delta x_i |F(x_i)-{\rm Exact}(x_i)|$.  These
numbers confirm the qualitative differences noted previously.

We also run the same simulation but with two levels of AMR (using a
refinement factor of 2), triggered based on a normalized slope greater
than 10\% in the density.  This refinement criterion results in only
the refinement of strong gradients, and does not include the
rarefaction fan at late times.  The results are shown in the bottom
row of Figure~\ref{fig.sodshocktube}.  Using AMR, the results are much
better for all three methods, with much sharper shocks and contact
discontinuities and even a better representation of the rarefaction
wave, which is only refined beyond the root grid at early times.
Although the results are improved for all methods, PPM still produces
the best result, as is shown clearly by the computed error norms
(displayed again in each of the individual panels).

\begin{figure}
\begin{center}
\includegraphics[width=\textwidth]{figures/SodShockTube.eps}
\caption{The density distribution of the classic Sod Shock Tube for
three different solvers (from left to right column) and with (bottom
row) and without (top row) AMR.  In each case 100 zones are used on
the root level and the results are shown at $t=0.25$.  All cells are
plotted and color-coded by level with purple indicating level 0, green
level 1 and red level 2 (at the time shown, only small region
surrounding the contact discontinuity and the shock are refined).  In
each panel, we show the analytic solution as a solid line and the
$E_1$ error norm in the upper right.}
\label{fig.sodshocktube}
\end{center}
\end{figure}



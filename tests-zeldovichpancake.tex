\subsubsection{Zel'dovich Pancake}
\label{sec.tests.pancake}

The Zel'dovich pancake \citep{1970A&A.....5...84Z} is particularly
relevant to cosmological simulations, because it includes many
features that are critical to structure formation: hydrodynamics,
expansion, and self gravity.  This problem represents the formation of an ideal,
isolated caustic, and is thus a useful proxy for much more
complicated structures in full 3-dimensional cosmological simulations,
such as the collapse of gas onto a cosmological halo or filament.

The initial conditions are simple, and we follow the prescription of
\citet{Anninos94}.  Assuming a geometrically flat cosmology, the density
perturbation is given by
\begin{equation}
\rho(x_l) = \rho_0 \left[ 1 - \frac{1+z_c}{1+z} \cos(k x_l) \right]
\end{equation}
with the internal energy of the gas set so that the entropy 
is constant throughout.  The velocity perturbation is given by
\begin{equation}
v(x_l) = -H_0 \frac{1 + z_c}{(1+z)^{1/2}} \frac{\sin(k x_l)}{k}
\end{equation}

In the equations above, $\rho_0$ is the background density, z$_c$ is a
free parameter and is the redshift where the sheet forms a caustic
(i.e., where it `pancakes'), z is the redshift of initialization, x$_l$ is the
Lagrangian mass coordinate, $k = 2 \pi / \lambda$ (where $\lambda$ is
the perturbation wavelength), and H$_0$ is the value of the Hubble
constant at $z = 0$.  Note that this solution is expressed in terms of 
Lagrangian positions, so one needs to convert this into the Eulerian
coordinates, x$_e$, that are more useful to a grid-based calculation:
\begin{equation}
x_e = x_l - \frac{1 + z_c}{1 + z} \frac{\sin(k x_l)}{k}
\end{equation}

We note that the solution described above is exact up to the point of
caustic formation.  In Figure~\ref{fig.pancake}, we show the results
of a test of the adaptive mesh version of \enzo's Zel'dovich Pancake
test.  A one-dimensional box of length $64$~Mpc/h is initialized at $z
= 20$ in an $\Omega_M = 1$ universe with $h = 0.5$ and a background
temperature of 100~K, with a background density of $\rho_0 = \rho_c$.
The simulation is initialized with 64 grid cells, refining by factors
of four using criteria based on cell mass and the presence of shocks,
for a maximum of 2 levels (i.e., an equivalent maximum resolution of
1024 grid cells).  The simulation is evolved to z$ = 0$ using the PPM
hydro method.

The final output of the calculation, with the key features of this
test problem, is shown in Figure~\ref{fig.pancake}.
The strong shocks and large density gradients are well-resolved, with
density and velocity jumps being well-delineated and at the correct
locations.  The key features of 
this test problem can be resolved with far fewer cells -- simulations
including a mere 8 cells resolve the key features, as shown in
Sections 3.3.4-3.3.5 of~\citet{BryanThesis96} -- but we choose a
higher resolution here for illustrative purposes.

\begin{figure}
\begin{center}
\includegraphics[width=0.8\textwidth]{figures/AMRZeldovichPancake.eps}
\caption{Zel`Dovich Pancake test shown at z$ = 0$, initialized in a
$\Omega_m = 1$ universe at z$ = 20$ on a one-dimensional grid having
64 cells, and further refined by factors of four based on cell mass
and the presence of shocks for up to two additional levels of mesh,
having a maximum effective resolution of 1024 grid cells. The top,
middle, and bottom rows show density, temperature, and velocity of the
gas, respectively; all are a function of position in units of the box
size.  The central region ($x \simeq 0.45-0.55$) has been adaptively
refined, as can be seen by the locations of grid points.  Shocks and
the central density peak are clearly resolved, with well-delineated
jumps at the appropriate locations.}
\label{fig.pancake}
\end{center}
\end{figure}

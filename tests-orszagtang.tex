\subsubsection{MHD: Orszag-Tang Vortex}
\label{sec.tests.mhd}

Figure~\ref{fig.orszag} shows the Orszag-Tang vortex problem
\citep{Orszag79}.  The left panel shows the result using \enzo's
constrained transport MHD method, while the right panel shows the
result using \enzo's implementation of Dedner MHD.  The Orszag-Tang
vortex test is a classig MHD test problem, and shows that significant
small scale structure can be generated in MHD from large scale initial
perturbations.  It is often used to compare the effective resolution
of different MHD schemes.  The test begins with uniform density,
$\rho_0=25/(36\pi)$, and pressure, $P_0=5/(12\pi)$ (as with other
tests in this section, in the absence of gravity or chemistry/cooling
we use dimensionless units).  There is a single rotational mode in the
velocity, and two in the magnetic field: ${\bf v}_0 = (-\sin(2\pi y)
\hat{x},\, \sin(2\pi x) \hat{y})$, ${\bf B}_0 = (-\sin(2\pi y)
\hat{x},\, \sin(4\pi x) \hat{y})$.  The simulation is evolved to
$t=0.48$.  One can see that the structures are accurately represented
as compared to, for example, \citet{Toth00}, and that the resolution
of shocks is comparable in both methods.

\begin{figure}
\begin{center}
\includegraphics[width=0.4\textwidth]{figures/MHDCT_OrszagTang_Density.eps}
\includegraphics[width=0.4\textwidth]{figures/MHDDedner_OrszagTang_Density.eps}
\caption{Density field from the Orszag-Tang vortex test, at $t=0.48$.
Left: solution using constrained transport MHD.  Right: solution using
Dedner MHD. The initial conditions are uniform density, with a single
rotating velocity structure and two circular magnetic structures.
These initial conditions generate significant small-scale structure in
both the CT and Dedner schemes, which have approximately equal
effective resolution.}
\label{fig.orszag}
\end{center}
\end{figure}
